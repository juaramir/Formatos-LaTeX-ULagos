\chapter{Contenidos de ejemplo}
a continuación te doy ejemplos de figuras, formular y tablas

\section{Figuras}

\blindtext %reemplazar esta linea


\begin{figure}[H]
\centering
 \includegraphics[scale=0.1]{portada}
  \caption{Organigrama del área \dots}
\end{figure}


\begin{figure}[H]
\centering
 \includegraphics[scale=0.1]{portada}
  \caption{Organigrama de la empresa \dots}
\end{figure}

\section{Tablas}

\blindtext %reemplazar esta linea

\begin{table}[H]
\centering
\begin{tabular}{|c|c|c|}\hline
  A&B &C \\
  \hline
  D&E &F \\
 G & H& I\\\hline
\end{tabular}
\caption{Ejemplo Tabla texto centrado}
\label{t:01}
\end{table}

\blindtext %reemplazar esta linea
 \ref{t:02}

\begin{table}[H]
\centering
\begin{tabular}{|l|c|r|}\hline
  A&B &C \\
  \hline
  D&E &F \\
 G & H& I\\\hline
\end{tabular}
\caption{Ejemplo Tabla texto en todos los ordenes}
\label{t:02}
\end{table}

\blindtext %reemplazar esta linea

\begin{table}[H]
\centering
\begin{tabular}{|p{3cm}|p{5cm}|p{2cm}|}\hline
  A&B &C \\
  \hline
  D&E &F \\
 G & H& I\\\hline
\end{tabular}
\caption{Ejemplo Tabla con tamaño de columnas predefinidos}
\label{t:03}
\end{table}
\section{Formulas}

\blindtext %reemplazar esta linea


\begin{displaymath}
C_L=\frac{(S_{22}-\Delta S_{11}^*)^*}{|S_{22}|^2=-|\Delta|^2}
\end{displaymath}
    
\begin{displaymath}
R_S=\frac{\sqrt{1-g_s}\cdot (1-|S_{11}|^2)}{1-(1-g_s)\cdot|S_{11}|^2}
\end{displaymath}


\section{Ámbitos}
Ejemplos de algunos ámbitos para remarcar las cosas, si es que es una definición, ejemplo, demostración, entre otros.
\subsection{Lorem}
\blindtext %reemplazar esta linea

\begin{definir}\label{def:01}
\blindtext %reemplazar esta linea
\end{definir}



\subsubsection{Ámbitos y referencias}

    \blindtext %reemplazar esta linea
\ref{dem:01}.
\begin{demo}\label{dem:01}
\blindtext %reemplazar esta linea
\end{demo}

    Lorem ipsum dolor sit amet, consectetur adipiscing elit, sed do eiusmod tempor incididunt ut labore et dolore magna aliqua, como se ve en el Ejemplo \ref{ej:a}.

\begin{ejemplo}\label{ej:a}
    \blindtext %reemplazar esta linea

\end{ejemplo}

    Lorem ipsum dolor sit amet, consectetur adipiscing elit, sed do eiusmod tempor incididunt ut labore et dolore magna aliqua, como se ve en la Prueba \ref{pru:01}.
\begin{prueba}\label{pru:01}
   \blindtext %reemplazar esta linea

\end{prueba}


    Lorem ipsum dolor sit amet, consectetur adipiscing elit, sed do eiusmod tempor incididunt ut labore et dolore magna aliqua, como se ve en la Observación \ref{ob:01}.
\begin{obs}\label{ob:01}
    Lorem ipsum dolor sit amet, consectetur adipiscing elit, sed do eiusmod tempor incididunt ut labore et dolore magna aliqua.
\end{obs}


\section{Algoritmos}\label{A:alg}
\blindtext %reemplazar esta linea

Lorem ipsum dolor sit amet, consectetur adipiscing elit, sed do eiusmod tempor incididunt ut labore et dolore magna aliqua, como en el Algoritmo \ref{CodC}.

\lstset{language=C}
\begin{lstlisting}[caption = C\'odigo en C de una sumatoria, label = CodC]
#include <stdio.h>
#include <stdlib.h>
/* Algoritmo para realizar la sumatoria */
/* S=2+4+6+...+2n */

int main(void){
	int i,s,n;
	
	/* inicializar el valor de la sumatoria en 0 */
	s=0;
	printf("ingrese la cantidad de elementos de la sumatoria=");
	scanf("% d", &n);
	/* Realiza la iteracion n veces, y el indice "i" lo multiplica por */
	/* 2 y lo va sumando a s*/
	for(i=1;i<=n;i++){
		s = s	+ 2*i;
	} 
	printf("el resultado de la sumatoria es=% d\n",s);

	return (0);
}
\end{lstlisting}


Lorem ipsum dolor sit amet, consectetur adipiscing elit, sed do eiusmod, en el Algoritmo \ref{codL} tempor incididunt ut labore et dolore magna aliqua.

\lstset{language=LISP}
\begin{lstlisting}[caption= C\'odigo LISP de una Lista, label = codL]
(define (length x)
    (if (list? x) (length-aux x)
        (error "x no es una lista")))
        
(define (length-aux x)
    (if (null? x) 0 (+1 (length-aux (cdr x)))))
\end{lstlisting}

Lorem ipsum dolor sit amet, consectetur adipiscing elit, sed do eiusmod tempor incididunt ut, en el Algoritmo \ref{codP} labore et dolore magna aliqua.


\lstset{language=PROLOG}
\begin{lstlisting}[caption= C\'odigo PROLOG de un \'arbol geneal\'ogico, label=codP]
% Arbol genealogico version 1.
% padre(A,B) significa que B es el padre de A.

padre(juan,alberto).
padre(luis,alberto).
padre(alberto,leoncio). 
padre(geronimo,leoncio).
padre(luisa,geronimo). 

% Ahora se define las condiciones para que dos individuos sean hermanos hermano(A,B), significa que A es hermano de B.
hermano(A,B) :- 
    padre(A,P), 
    padre(B,P), 
    A \== B.
% Ahora se define el parentesco abuelo-nieto.  nieto(A,B) significa que A es nieto de B.
nieto(A,B) :- 
    padre(A,P), 
    padre(P,B). 
\end{lstlisting}

Lorem ipsum dolor sit amet, consectetur adipiscing elit, sed do eiusmod tempor incididunt ut labore et dolore magna aliqua.

   \lstset{language=java}
\begin{lstlisting}[caption= C\'odigo JAVA de una clase]
class <Nombre>{
   public static void main(String[] args){
      instrucciones;
   }
}
\end{lstlisting}

\subsection{Ejemplo referencia a código en otro capitulo}
En el Anexo \ref{A:01} puedes encontrar el Algoritmo \ref{codPH} el cual es un ejemplo de código PHP.



\section{Uso Bibliografia}


Para usar la bibliografía de un autor lo cito \cite{DBLPMH10}, si quiero mencionar a mas de una documento los cito  así \cite{ABC02,JanChomicki2008,SN01}


\chapter{Generalidades}

Es lo que comúnmente se conoce como introducción, conduce al lector desde un tema de un área general hacia un campo de investigación específico, describe el contexto, el problema, motiva al lector.

Introduce la terminología, destaca las contribuciones del documento y da una breve descripción de la organización de éste.

Ejemplo de uso de una referencia \cite{SN01}. Ejemplo de referencia doble \cite{ABC02,JanChomicki2008}.


\section{Origen del Tema}
Contextualiza el trabajo respecto de investigaciones previas de otros autores y propias, señala las diferencias con trabajos previos. Algunas veces se incluye en la introducción o bien en la discusión del trabajo (secciones finales). Largo aproximado: 2 páginas.
\section{Planteamiento}

Provee un \naranjo{marco de referencia} para interpretar los resultados y conectarlos a la literatura existente sobre el fenómeno, orienta sobre cómo se realizará el estudio.

 Ayuda a prevenir errores que se han cometido en otros estudios, conduce al establecimiento de la hipótesis o afirmaciones que se someterán a prueba.
 
 Amplia el horizonte del estudio y centra al investigador en el problema, para evitar desviaciones del planteamiento original.

Considera una \naranjo{revisión bibliográfica} que consiste en detectar, obtener y consultar la bibliografía y otros materiales que pueden ser útiles para los propósitos del estudio.

La revisión bibliográfica debe ser selectiva, se puede realizar a partir de tres fuentes principales:

\begin{itemize}\justifying
  \item \naranjo{Primarias (directas):} Libros, artículos, antologías, tesis, disertaciones, entre otros.
  \item \naranjo{Secundarias:} Compilaciones, resúmenes de listados de referencias publicadas en un área en particular, bases de datos.
  \item \naranjo{Terciarias:} Documentos que reúnen nombres y títulos de revistas y otras publicaciones.
\end{itemize}



\begin{ejemplo}
\blindtext %reemplazar esta linea
\end{ejemplo} 

\section{Justificación y Aporte}
Justificar la conveniencia del proyecto desde diversos puntos de vista.

Preguntas clave:
  \begin{itemize}
  \item ¿Para qué sirve la investigación?
  \item ¿Quiénes se benefician con los resultados?
  \item ¿Ayuda a resolver algún problema práctico?
  \item ¿Contribuye a aumentar el conocimiento?
  \item ¿Se podrán generalizar los resultado?
\end{itemize}



\begin{ejemplo}
\blindtext %reemplazar esta linea
\end{ejemplo}


\section{Viabilidad}
Analizar la disponibilidad de recursos financieros, humanos y materiales.

Preguntas clave:
  \begin{itemize}\justifying
  \item ¿Puede llevarse a cabo esta investigación?
  \item ¿Cuánto tiempo tomará realizarla?
\end{itemize}

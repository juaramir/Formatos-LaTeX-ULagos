\documentclass[letter,12pt]{report}
\usepackage[spanish,es-tabla]{babel}
\usepackage[sfdefault, condensed]{roboto}
\usepackage[utf8]{inputenc}
\usepackage[T1]{fontenc}
\usepackage[margin=1cm]{geometry}
\usepackage{multicol,graphicx,fancyhdr,eso-pic,url,float,cite,lmodern,listings,times,textcomp,amsthm,amsmath,amssymb,dsfont,color,colortbl,sidecap,xspace,epic,eepic,anysize, setspace}
\usepackage{lipsum}
%\usepackage{apacite}
\usepackage{blindtext}

  \providecommand{\keywords}[1]{\textbf{\textit{Palabras Clave---}} #1}



%Tipos de Letra
%\renewcommand{\rmdefault}{phv} % Arial
\usepackage{mathptmx} %Times
%Margenes
\marginsize{3cm}{2cm}{2cm}{2cm}
\spacing{1.5}%interlineado


%%%Entornos de desarrollo
\newtheorem{ejemplo}{Ejemplo}
\newtheorem{definir}{Definición}
\newtheorem{prueba}{Prueba}
\newtheorem{demo}{Demostración}
\newtheorem{obs}{Observación}

\newcommand\BackgroundPic{ \put(-3,0){ \parbox[b][\paperheight]{\paperwidth}{ \vfill \centering \includegraphics[width=\paperwidth,height=\paperheight]{portada.jpg} \vfill }}} 
  
  %Definicion de Colores
\definecolor{gray97}{gray}{.97}
\definecolor{gray75}{gray}{.75}
\definecolor{gray45}{gray}{.45}
\definecolor{verdeo}{rgb}{0,.5,0.2}
\definecolor{listinggray}{gray}{0.9}
\definecolor{lbcolor}{rgb}{0.9,0.9,0.9}
\newcommand\gris[1]{\textcolor[gray]{.35}{\emph{#1}}}
\newcommand\rojo[1]{\textcolor[rgb]{1,0,0}{#1}}
\newcommand\blue[1]{\textcolor[rgb]{0,0,1}{{#1}}}
\newcommand\azul[1]{\textcolor[rgb]{0,0,1}{#1}}
\newcommand\verde[1]{\textcolor[rgb]{0,.5,0.2}{#1}}
\newcommand\naranjo[1]{\textcolor[rgb]{1.00,0.36,0.06}{\textbf{#1}}}
\newcommand\blanco[1]{\textcolor[rgb]{1,1,1}{\textbf{#1}}}
\newcommand {\red}[1]{\textcolor[rgb]{1.00,0.00,0.00}{#1}}
\newcommand\cita[1]{{\scriptsize \begin{flushright}\emph{(#1)}\end{flushright}}}
  
  %%%CODIGOS DE PROGRAMACION
    \renewcommand{\lstlistingname}{Algoritmo}
\lstset{
	tabsize=4,
	rulecolor=,
	language=C,
        basicstyle=\footnotesize,
        upquote=true,
        aboveskip={1.5\baselineskip},
        columns=fixed,
        showstringspaces=false,
        extendedchars=true,
        breaklines=true,
        prebreak = \raisebox{0ex}[0ex][0ex]{\ensuremath{\hookleftarrow}},
        frame=single,
        showtabs=false,
        showspaces=false,
        showstringspaces=false,
        identifierstyle=\ttfamily,
        numbers=left,
        numbersep=15pt,
        numberstyle=\tiny,
        keywordstyle=\bfseries \color[RGB]{0,2,216},
        commentstyle=\color[rgb]{0,.5,0.2},
        stringstyle=\color[RGB]{216,0,114},
}
%%%%FIN CODIGOS DE PROGRAMACION
\def\figurename{}
  
%%%%%%%%%%ENCABEZADO Y PIE DE PAGINA
%encabezado de las paginas pares e impares.
\rhead[PP]{Ing. Civil en Informática}
\renewcommand{\headrulewidth}{0.5pt}
%pie de pagina de las paginas pares e impares.
\lfoot[nombre]{Nombre Apellido}
\rfoot[rut]{Universidad de los Lagos}
\renewcommand{\footrulewidth}{0.5pt}
%encabezado y pie de pagina de la pagina inicial de un capitulo.
\fancypagestyle{plain}{
\fancyhead[R]{Ing. Civil en Informática}
\fancyfoot[L]{Nombre Apellido}
\fancyfoot[R]{Universidad de los Lagos}
\renewcommand{\headrulewidth}{0.5pt}
\renewcommand{\footrulewidth}{0.5pt}
}
\pagestyle{fancy} 
%%%%%%%%%%FIN ENCABEZADO Y PIE DE PAGINA 
 
 
 
 
\begin{document}
%%%%%%%%%%%PORTADA%%%%%%%%%%%%%%%%%%%%%
\setlength{\unitlength}{1 cm} %Especificar unidad de trabajo
\thispagestyle{empty}

\AddToShipoutPicture*{\BackgroundPic}

   \title{\vspace{-2cm}\scshape\Huge{Título}\\\vspace{1cm}
        \Large Departamento de Ciencias Exactas\\
        \Large Ingeniería Civil en Informática\\\large Osorno, Chile}
   \author{
      Autor 1\\
      Correo 1
      \and Autor 2\\
      Correo 2}
    
   \date{Profesor Guía: Nombre Profesor\\\
   Profesor Co-guía: Nombre Profesor\\\
   \today}
   \maketitle
   \ClearShipoutPicture
%%%%       INICIO DEL DOCUMENTO
\cleardoublepage
\pagenumbering{roman}
\setcounter{page}{1}

%%%%Agradecimientos
\chapter*{Agradecimientos}
Este trabajo no habría sido posible sin el apoyo y el estímulo de

 $\dots$ \blindtext %reemplazar esta linea

%%%%%%%%RESUMEN
\begin{abstract}
Resume en un (1) párrafo el contenido del informe en un máximo de 350 palabras.
Debe ser preciso:
\begin{itemize}\justifying
  \item Establece el problema
  \item Dice porqué es interesante
  \item Señala los logros y desafios
\end{itemize}
Un resumen debe ser llamativo, motivador, descriptivo y sin contenido específico. \textbf{No incluye}: citas, referencias, conclusiones, figuras ni tablas.


\keywords{Palabra1, Palabra2, Palabra3, Palabra4, Palabra5}
\end{abstract}


%%%%%%INDICES
\tableofcontents
\listoffigures
\renewcommand{\listtablename}{Índice de tablas}
\listoftables
\renewcommand{\lstlistlistingname}{Índice de Algoritmos}
\lstlistoflistings
%\addcontentsline

\cleardoublepage
\pagenumbering{arabic}
\setcounter{page}{1}

%##########CONTENIDO
\chapter{Generalidades}

Es lo que comúnmente se conoce como introducción, conduce al lector desde un tema de un área general hacia un campo de investigación específico, describe el contexto, el problema, motiva al lector.

Introduce la terminología, destaca las contribuciones del documento y da una breve descripción de la organización de éste.

Ejemplo de uso de una referencia \cite{SN01}. Ejemplo de referencia doble \cite{ABC02,JanChomicki2008}.


\section{Origen del Tema}
Contextualiza el trabajo respecto de investigaciones previas de otros autores y propias, señala las diferencias con trabajos previos. Algunas veces se incluye en la introducción o bien en la discusión del trabajo (secciones finales). Largo aproximado: 2 páginas.
\section{Planteamiento}

Provee un \naranjo{marco de referencia} para interpretar los resultados y conectarlos a la literatura existente sobre el fenómeno, orienta sobre cómo se realizará el estudio.

 Ayuda a prevenir errores que se han cometido en otros estudios, conduce al establecimiento de la hipótesis o afirmaciones que se someterán a prueba.
 
 Amplia el horizonte del estudio y centra al investigador en el problema, para evitar desviaciones del planteamiento original.

Considera una \naranjo{revisión bibliográfica} que consiste en detectar, obtener y consultar la bibliografía y otros materiales que pueden ser útiles para los propósitos del estudio.

La revisión bibliográfica debe ser selectiva, se puede realizar a partir de tres fuentes principales:

\begin{itemize}\justifying
  \item \naranjo{Primarias (directas):} Libros, artículos, antologías, tesis, disertaciones, entre otros.
  \item \naranjo{Secundarias:} Compilaciones, resúmenes de listados de referencias publicadas en un área en particular, bases de datos.
  \item \naranjo{Terciarias:} Documentos que reúnen nombres y títulos de revistas y otras publicaciones.
\end{itemize}



\begin{ejemplo}
\blindtext %reemplazar esta linea
\end{ejemplo} 

\section{Justificación y Aporte}
Justificar la conveniencia del proyecto desde diversos puntos de vista.

Preguntas clave:
  \begin{itemize}
  \item ¿Para qué sirve la investigación?
  \item ¿Quiénes se benefician con los resultados?
  \item ¿Ayuda a resolver algún problema práctico?
  \item ¿Contribuye a aumentar el conocimiento?
  \item ¿Se podrán generalizar los resultado?
\end{itemize}



\begin{ejemplo}
\blindtext %reemplazar esta linea
\end{ejemplo}


\section{Viabilidad}
Analizar la disponibilidad de recursos financieros, humanos y materiales.

Preguntas clave:
  \begin{itemize}\justifying
  \item ¿Puede llevarse a cabo esta investigación?
  \item ¿Cuánto tiempo tomará realizarla?
\end{itemize}

\chapter{Proyecto de Título}
Una \naranjo{hipótesis} es cualquier suposición, conjetura o predicción que se base en conocimientos existentes, en \naranjo{hechos nuevos} o en ambos.

 Propone una respuesta anticipada del problema, por lo que será aceptada o rechazada como resultado de la investigación.

\begin{itemize}
  \item No puede contradecir hechos conocidos y comprobados con anterioridad.
  \item Debe ser factible de comprobación.
  \item Tiene que ofrecer una explicación suficiente de los hechos o condiciones que pretende abarcar.
  \item \naranjo{Tiene que estar relacionada con el sistema de conocimiento correspondiente a los hechos que plantea el problema}.
\end{itemize}

Las hipótesis se pueden clasificar por:
 \begin{itemize}
  \item Contenido
  \item Forma
  \item Procesamiento estadístico
\end{itemize}

Hipótesis por Contenido
\begin{itemize}
  \item \textbf{Descriptivas}:
  \begin{itemize}
  \item Plantean supuestos sobre la estructura, las manifestaciones, funciones y características del objeto estudiado.
\end{itemize}

  \item \textbf{Causales o explicativas}:
  \begin{itemize}
  \item Plantean supuestos acerca de los vínculos de causa y efecto en el objeto estudiado que requieren comprobación experimental.
\end{itemize}

\end{itemize}


Hipótesis por Forma
\begin{itemize}
  \item \textbf{Generales}:
  \begin{itemize}
  \item Plantean supuestos que poseen un carácter generalizado del objeto de investigación.
\end{itemize}

  \item \textbf{Operacionales}:
  \begin{itemize}
  \item Plantean supuestos que en el desarrollo de la investigación tendrán carácter provisional hasta que se demuestre lo contrario.
\end{itemize}

\end{itemize}

Hipótesis por Procesamiento Estadístico
\textbf{Alternativa y nula}
 \begin{itemize}\justifying
  \item Utilizadas en estadística inferencial.
  \item Se plantean hipótesis mutuamente excluyentes: \naranjo{hipótesis nula} o \naranjo{hipótesis de investigación}.
  \item La hipótesis de investigación es una afirmación especial cuya validez se pretende demostrar, si las pruebas empíricas no la apoyan, se aceptará la hipótesis nula y se abandona la hipótesis.
\end{itemize}

\begin{ejemplo}
  Es posible optimizar la evaluación de consultas conjuntivas en bases de datos a través de índices.
\end{ejemplo}

\begin{ejemplo}
  Es posible crear un modelo de calidad de datos para portales web, que pueda ser usado para evaluar el nivel de calidad de los datos provistos por un portal web.
\end{ejemplo}

\section{Objetivos}\label{objetivos}

Se deben abordar desde el principio de la investigación, expresan los fines que se esperan lograr con el estudio del problema planteado, responden a la pregunta \naranjo{¿Para qué se lleva a cabo la investigación?}, por lo general comienzan con un verbo en infinitivo: Determinar, identificar, establecer, distinguir, medir, cuantificar, entre otros.

Deben enunciar un resultado unívoco, preciso, factible y medible. Su formulación debe ser clara, concisa y bien orientada hacia el fin, en función de ellos se plantean los métodos de recolección de datos, pruebas estadísticas, entre otros.

Evitar unir objetivos, idealmente, un objetivo general y varios específicos.

Cada objetivo específico se ``mapea'' a una pregunta de investigación.
Por ejemplo:
  \begin{itemize}
  \item \textbf{\naranjo{Objetivo:}} Optimizar los métodos de acceso a disco.
  \item \textbf{\naranjo{Preguntas de investigación:}} ¿Cuáles son los métodos de acceso a disco?
\end{itemize}

\subsection{General}
\blindtext %reemplazar esta linea

\subsection{Específicos}
\begin{enumerate}\justifying
  \item 
  \blindtext %reemplazar esta linea

  \item 
  \blindtext %reemplazar esta linea


\end{enumerate}

\section{Alcance}
Que se planea realizar y hasta que punto se espera llegar.

Esta subdivisión debe:
\begin{enumerate}\justifying
  \item Identifique el producto del software para ser diseñado por el nombre (por ejemplo, Anfitrión DBMS, el Generador del Reporte, etc.);
  \item Explique eso que el producto (del software hará y que no hará.
  \item Describe la aplicación del software especificándose los beneficios pertinentes, objetivos, y metas;
  \item Sea consistente con las declaraciones similares en las especificaciones de niveles superiores (por ejemplo, las especificaciones de los requisitos del sistema), si ellos existen.
\end{enumerate}


\section{Metodología}
Esto no es hacer referencia a métodos y herramientas que se usarán en el desarrollo del trabajo. Sino que describir como se llevará a cabo el trabajo.

Por lo tanto, nuevamente se puede plantear la solución (el proyecto) en términos explícitos de: los objetivos generales y específicos.

Posteriormente relacionar el cumplimiento de los objetivos específicos con tareas o actividades a desarrollar (al final se debe incluir seguramente actividades de validación y prueba del producto - plan de prueba).


\subsection{Planificación y Carta Gantt}
\blindtext %reemplazar esta linea


\chapter{Resumen del Proyecto}
\section{Definición del Problema}
\blindtext %reemplazar esta linea
 

\section{Conceptos Previos}
Generar un resumen de todos los conceptos y tecnologías utilizadas y relacionadas para que el lector entienda más adelante de lo que se habla.
\section{Estado del Arte}
Trabajo previo (si corresponde) y los proyectos que respaldan la realización del proyecto de título.
\chapter{Desarrollo}
\blindtext %reemplazar esta linea

\section{Introducción}
\blindtext %reemplazar esta linea

\section{Lenguaje de Programación Elegido}
\blindtext %reemplazar esta linea

\subsection{Propiedades}
\blindtext %reemplazar esta linea

\subsection{Carencias}
\blindtext %reemplazar esta linea

\subsection{Ventajas}
\blindtext %reemplazar esta linea

\subsection{Inconvenientes}
\blindtext %reemplazar esta linea

\section{Definición del Problema}
\blindtext %reemplazar esta linea

\section{Propuesta de Solución}
\blindtext %reemplazar esta linea
\chapter{Pruebas}
\blindtext %reemplazar esta linea

\chapter{Contenidos de ejemplo}
a continuación te doy ejemplos de figuras, formular y tablas

\section{Figuras}

\blindtext %reemplazar esta linea


\begin{figure}[H]
\centering
 \includegraphics[scale=0.1]{portada}
  \caption{Organigrama del área \dots}
\end{figure}


\begin{figure}[H]
\centering
 \includegraphics[scale=0.1]{portada}
  \caption{Organigrama de la empresa \dots}
\end{figure}

\section{Tablas}

\blindtext %reemplazar esta linea

\begin{table}[H]
\centering
\begin{tabular}{|c|c|c|}\hline
  A&B &C \\
  \hline
  D&E &F \\
 G & H& I\\\hline
\end{tabular}
\caption{Ejemplo Tabla texto centrado}
\label{t:01}
\end{table}

\blindtext %reemplazar esta linea
 \ref{t:02}

\begin{table}[H]
\centering
\begin{tabular}{|l|c|r|}\hline
  A&B &C \\
  \hline
  D&E &F \\
 G & H& I\\\hline
\end{tabular}
\caption{Ejemplo Tabla texto en todos los ordenes}
\label{t:02}
\end{table}

\blindtext %reemplazar esta linea

\begin{table}[H]
\centering
\begin{tabular}{|p{3cm}|p{5cm}|p{2cm}|}\hline
  A&B &C \\
  \hline
  D&E &F \\
 G & H& I\\\hline
\end{tabular}
\caption{Ejemplo Tabla con tamaño de columnas predefinidos}
\label{t:03}
\end{table}
\section{Formulas}

\blindtext %reemplazar esta linea


\begin{displaymath}
C_L=\frac{(S_{22}-\Delta S_{11}^*)^*}{|S_{22}|^2=-|\Delta|^2}
\end{displaymath}
    
\begin{displaymath}
R_S=\frac{\sqrt{1-g_s}\cdot (1-|S_{11}|^2)}{1-(1-g_s)\cdot|S_{11}|^2}
\end{displaymath}


\section{Ámbitos}
Ejemplos de algunos ámbitos para remarcar las cosas, si es que es una definición, ejemplo, demostración, entre otros.
\subsection{Lorem}
\blindtext %reemplazar esta linea

\begin{definir}\label{def:01}
\blindtext %reemplazar esta linea
\end{definir}



\subsubsection{Ámbitos y referencias}

    \blindtext %reemplazar esta linea
\ref{dem:01}.
\begin{demo}\label{dem:01}
\blindtext %reemplazar esta linea
\end{demo}

    Lorem ipsum dolor sit amet, consectetur adipiscing elit, sed do eiusmod tempor incididunt ut labore et dolore magna aliqua, como se ve en el Ejemplo \ref{ej:a}.

\begin{ejemplo}\label{ej:a}
    \blindtext %reemplazar esta linea

\end{ejemplo}

    Lorem ipsum dolor sit amet, consectetur adipiscing elit, sed do eiusmod tempor incididunt ut labore et dolore magna aliqua, como se ve en la Prueba \ref{pru:01}.
\begin{prueba}\label{pru:01}
   \blindtext %reemplazar esta linea

\end{prueba}


    Lorem ipsum dolor sit amet, consectetur adipiscing elit, sed do eiusmod tempor incididunt ut labore et dolore magna aliqua, como se ve en la Observación \ref{ob:01}.
\begin{obs}\label{ob:01}
    Lorem ipsum dolor sit amet, consectetur adipiscing elit, sed do eiusmod tempor incididunt ut labore et dolore magna aliqua.
\end{obs}


\section{Algoritmos}\label{A:alg}
\blindtext %reemplazar esta linea

Lorem ipsum dolor sit amet, consectetur adipiscing elit, sed do eiusmod tempor incididunt ut labore et dolore magna aliqua, como en el Algoritmo \ref{CodC}.

\lstset{language=C}
\begin{lstlisting}[caption = C\'odigo en C de una sumatoria, label = CodC]
#include <stdio.h>
#include <stdlib.h>
/* Algoritmo para realizar la sumatoria */
/* S=2+4+6+...+2n */

int main(void){
	int i,s,n;
	
	/* inicializar el valor de la sumatoria en 0 */
	s=0;
	printf("ingrese la cantidad de elementos de la sumatoria=");
	scanf("% d", &n);
	/* Realiza la iteracion n veces, y el indice "i" lo multiplica por */
	/* 2 y lo va sumando a s*/
	for(i=1;i<=n;i++){
		s = s	+ 2*i;
	} 
	printf("el resultado de la sumatoria es=% d\n",s);

	return (0);
}
\end{lstlisting}


Lorem ipsum dolor sit amet, consectetur adipiscing elit, sed do eiusmod, en el Algoritmo \ref{codL} tempor incididunt ut labore et dolore magna aliqua.

\lstset{language=LISP}
\begin{lstlisting}[caption= C\'odigo LISP de una Lista, label = codL]
(define (length x)
    (if (list? x) (length-aux x)
        (error "x no es una lista")))
        
(define (length-aux x)
    (if (null? x) 0 (+1 (length-aux (cdr x)))))
\end{lstlisting}

Lorem ipsum dolor sit amet, consectetur adipiscing elit, sed do eiusmod tempor incididunt ut, en el Algoritmo \ref{codP} labore et dolore magna aliqua.


\lstset{language=PROLOG}
\begin{lstlisting}[caption= C\'odigo PROLOG de un \'arbol geneal\'ogico, label=codP]
% Arbol genealogico version 1.
% padre(A,B) significa que B es el padre de A.

padre(juan,alberto).
padre(luis,alberto).
padre(alberto,leoncio). 
padre(geronimo,leoncio).
padre(luisa,geronimo). 

% Ahora se define las condiciones para que dos individuos sean hermanos hermano(A,B), significa que A es hermano de B.
hermano(A,B) :- 
    padre(A,P), 
    padre(B,P), 
    A \== B.
% Ahora se define el parentesco abuelo-nieto.  nieto(A,B) significa que A es nieto de B.
nieto(A,B) :- 
    padre(A,P), 
    padre(P,B). 
\end{lstlisting}

Lorem ipsum dolor sit amet, consectetur adipiscing elit, sed do eiusmod tempor incididunt ut labore et dolore magna aliqua.

   \lstset{language=java}
\begin{lstlisting}[caption= C\'odigo JAVA de una clase]
class <Nombre>{
   public static void main(String[] args){
      instrucciones;
   }
}
\end{lstlisting}

\subsection{Ejemplo referencia a código en otro capitulo}
En el Anexo \ref{A:01} puedes encontrar el Algoritmo \ref{codPH} el cual es un ejemplo de código PHP.



\section{Uso Bibliografia}


Para usar la bibliografía de un autor lo cito \cite{DBLPMH10}, si quiero mencionar a mas de una documento los cito  así \cite{ABC02,JanChomicki2008,SN01}


%%%añade mas capitulos o secciones segun necesidades

\chapter{Conclusión}
Análisis de los objetivos propuestos/cumplidos y resumen de lo realizado.
\section{Principales aportes}
\section{Contraste de resultados}
\section{Trabajos futuros}
Solo si corresponde.

%%%%%%agregar referencias
\renewcommand{\refname}{Referencias}
\bibliographystyle{ieeetr}
\bibliography{bibliografia.bib}

%%%%%%ANEXOS DEL DOCUMENTO
\renewcommand{\appendixname}{Anexos}
\appendix

\chapter{Capitulo Anexo}
\section{Algoritmos desarrollados}\label{A:01}
Si en algún caso se elabora un software con líneas de código muy extensas, es recomendable incluirlas como anexo y referenciarlas que incluirlas en le mismo desarrollo.

\lstset{language=PHP}
\begin{lstlisting}[caption= C\'odigo PHP de impresi\'on de una variable, label = codPH]
<?php
    $v = 5;
    echo "El valor es: $v\n";
?>
\end{lstlisting}

\section{Titulo Sección}
Lorem ipsum dolor sit amet, consectetur adipiscing elit, sed do eiusmod tempor incididunt ut labore et dolore magna aliqua.

\end{document}

\chapter{Proyecto de Título}
\chapter{Hipotesis}
Una \naranjo{hipótesis} es cualquier suposición, conjetura o predicción que se base en conocimientos existentes, en \naranjo{hechos nuevos} o en ambos.

 Propone una respuesta anticipada del problema, por lo que será aceptada o rechazada como resultado de la investigación.

\begin{itemize}
  \item No puede contradecir hechos conocidos y comprobados con anterioridad.
  \item Debe ser factible de comprobación.
  \item Tiene que ofrecer una explicación suficiente de los hechos o condiciones que pretende abarcar.
  \item \naranjo{Tiene que estar relacionada con el sistema de conocimiento correspondiente a los hechos que plantea el problema}.
\end{itemize}

Las hipótesis se pueden clasificar por:
 \begin{itemize}
  \item Contenido
  \item Forma
  \item Procesamiento estadístico
\end{itemize}

Hipótesis por Contenido
\begin{itemize}
  \item \textbf{Descriptivas}:
  \begin{itemize}
  \item Plantean supuestos sobre la estructura, las manifestaciones, funciones y características del objeto estudiado.
\end{itemize}

  \item \textbf{Causales o explicativas}:
  \begin{itemize}
  \item Plantean supuestos acerca de los vínculos de causa y efecto en el objeto estudiado que requieren comprobación experimental.
\end{itemize}

\end{itemize}


Hipótesis por Forma
\begin{itemize}
  \item \textbf{Generales}:
  \begin{itemize}
  \item Plantean supuestos que poseen un carácter generalizado del objeto de investigación.
\end{itemize}

  \item \textbf{Operacionales}:
  \begin{itemize}
  \item Plantean supuestos que en el desarrollo de la investigación tendrán carácter provisional hasta que se demuestre lo contrario.
\end{itemize}

\end{itemize}

Hipótesis por Procesamiento Estadístico
\textbf{Alternativa y nula}
 \begin{itemize}\justifying
  \item Utilizadas en estadística inferencial.
  \item Se plantean hipótesis mutuamente excluyentes: \naranjo{hipótesis nula} o \naranjo{hipótesis de investigación}.
  \item La hipótesis de investigación es una afirmación especial cuya validez se pretende demostrar, si las pruebas empíricas no la apoyan, se aceptará la hipótesis nula y se abandona la hipótesis.
\end{itemize}

\begin{ejemplo}
  Es posible optimizar la evaluación de consultas conjuntivas en bases de datos a través de índices.
\end{ejemplo}

\begin{ejemplo}
  Es posible crear un modelo de calidad de datos para portales web, que pueda ser usado para evaluar el nivel de calidad de los datos provistos por un portal web.
\end{ejemplo}

\section{Objetivos}\label{objetivos}

Se deben abordar desde el principio de la investigación, expresan los fines que se esperan lograr con el estudio del problema planteado, responden a la pregunta \naranjo{¿Para qué se lleva a cabo la investigación?}, por lo general comienzan con un verbo en infinitivo: Determinar, identificar, establecer, distinguir, medir, cuantificar, entre otros.

Deben enunciar un resultado unívoco, preciso, factible y medible. Su formulación debe ser clara, concisa y bien orientada hacia el fin, en función de ellos se plantean los métodos de recolección de datos, pruebas estadísticas, entre otros.

Evitar unir objetivos, idealmente, un objetivo general y varios específicos.

Cada objetivo específico se ``mapea'' a una pregunta de investigación.
Por ejemplo:
  \begin{itemize}
  \item \textbf{\naranjo{Objetivo:}} Optimizar los métodos de acceso a disco.
  \item \textbf{\naranjo{Preguntas de investigación:}} ¿Cuáles son los métodos de acceso a disco?
\end{itemize}

\subsection{General}
\blindtext %reemplazar esta linea

\subsection{Específicos}
\begin{enumerate}\justifying
  \item 
  \blindtext %reemplazar esta linea

  \item 
  \blindtext %reemplazar esta linea


\end{enumerate}

\section{Alcance}
Que se planea realizar y hasta que punto se espera llegar.

Esta subdivisión debe:
\begin{enumerate}\justifying
  \item Identifique el producto del software para ser diseñado por el nombre (por ejemplo, Anfitrión DBMS, el Generador del Reporte, etc.);
  \item Explique eso que el producto (del software hará y que no hará.
  \item Describe la aplicación del software especificándose los beneficios pertinentes, objetivos, y metas;
  \item Sea consistente con las declaraciones similares en las especificaciones de niveles superiores (por ejemplo, las especificaciones de los requisitos del sistema), si ellos existen.
\end{enumerate}


\section{Metodología}
Esto no es hacer referencia a métodos y herramientas que se usarán en el desarrollo del trabajo. Sino que describir como se llevará a cabo el trabajo.

Por lo tanto, nuevamente se puede plantear la solución (el proyecto) en términos explícitos de: los objetivos generales y específicos.

Posteriormente relacionar el cumplimiento de los objetivos específicos con tareas o actividades a desarrollar (al final se debe incluir seguramente actividades de validación y prueba del producto - plan de prueba).


\subsection{Planificación y Carta Gantt}
\blindtext %reemplazar esta linea


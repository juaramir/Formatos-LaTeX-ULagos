\documentclass[aspectratio=43]{beamer}%[handout]
\mode<handout>{
\usepackage{pgfpages}
\pgfpagesuselayout{4 on 1}[letterpaper,border shrink=5mm, landscape]
}
\usepackage[T1]{fontenc}
\usepackage[utf8]{inputenc}
\usepackage[spanish]{babel}
\usepackage{beamerthemeshadow,beamerthemesplit,cite,cancel,lmodern,eso-pic,fancyvrb,textcomp,lmodern,url,times,booktabs,amssymb,amsmath,ragged2e,float,subfig,xspace,epic,eepic,multicol,multirow,colortbl,color,graphicx,url}
\usepackage[normalem]{ulem} %tachar texto con \sout{}
\usepackage{listings} %CODIGOs
\usepackage[sfdefault]{roboto}
\setcounter{tocdepth}{3}
\usepackage[figurename=]{caption}
%Colores Ulagos
\definecolor{gray97}{gray}{.97}
\definecolor{gray75}{gray}{.75}
\definecolor{gray45}{gray}{.45}
\definecolor{listinggray}{gray}{0.9}
\definecolor{lbcolor}{rgb}{0.9,0.9,0.9}
%Colores Ulagos
\definecolor{amarillo}{RGB}{255,182,18}
\definecolor{verde}{RGB}{52,178,51}
\definecolor{rojo}{RGB}{237,41,57}
\definecolor{azulu}{RGB}{19,15,204}
\definecolor{azul}{RGB}{1,110,185}
\definecolor{negro}{RGB}{35,31,32}
\definecolor{naranjo}{RGB}{251,79,20}
\newcommand\rojo[1]{\textcolor[RGB]{237,41,57}{#1}}
\newcommand\gris[1]{\textcolor[gray]{.65}{#1}}
\newcommand\azul[1]{\textcolor[RGB]{19,15,204}{#1}}
\newcommand\verde[1]{\textcolor[RGB]{5,101,99}{#1}}
\newcommand\naranjo[1]{\textcolor[RGB]{251,79,20}{#1}}

%%%%%%%%%%%%%%%%%%%%%%%%%
%% Tikz
%%%%%%%%%%%%%%%%%%%%%%%%%
\usepackage{tikz}
\usetikzlibrary{trees}
\usetikzlibrary{snakes}
\usetikzlibrary{arrows}
\usetikzlibrary{shapes}
\usetikzlibrary{backgrounds}
\usetikzlibrary{patterns}
\usetikzlibrary{fit}
\usetikzlibrary{positioning} % LATEX and plain TEX
%%%%%%%%%%%%%%%%%%%%%%%%%%

%%%%%%%%%%%%%%%%%%%%%TEMA
\mode<presentation>
\usetheme{CambridgeUS}
\usecolortheme[named=azul]{structure}
\useinnertheme{rectangles}    
\useoutertheme{infolines} 
               
\setbeamercovered{transparent}

\setbeamerfont{block title}{size={}}
\setbeamerfont{title}{shape=\scshape}
\setbeamerfont{frametitle}{shape=\scshape}
\setbeamerfont{author}{shape=\scshape}
\setbeamerfont{institute}{shape=\scshape}
\setbeamerfont{block title}{shape=\scshape}

\setbeamertemplate{blocks}[rounded][shadow=true]
\setbeamertemplate{itemize items}[default]
\setbeamertemplate{enumerate items}[circle]
\setbeamertemplate{description item}[align left]
\setbeamertemplate{blocks}[rounded][shadow=true]
\setbeamertemplate{title page}[default][colsep=-4bp,rounded=false,shadow=false]
%\setbeamertemplate{frametitle continuation}

\setbeamercolor*{palette primary}{use=structure,fg=azul,bg=listinggray}
\setbeamercolor*{palette secondary}{use=structure,fg=azul,bg=gray97}
\setbeamercolor*{palette tertiary}{use=structure,fg=gray97,bg=azul}
\setbeamercolor*{palette sidebar primary}{use=structure,fg=azul}
\setbeamercolor*{palette sidebar tertiary}{use=structure,fg=azul}
\setbeamercolor*{title}{use=structure,fg=white}
\setbeamercolor*{author}{use=structure,fg=white}
\setbeamercolor*{institute}{use=structure,fg=white}
\setbeamercolor{frametitle right}{bg=azul!50!white}
\setbeamercolor{structure}{fg=azul}
\setbeamercolor{block title}{use=structure,fg=white,bg=amarillo}
\setbeamercolor{block body}{use=structure,fg=negro,bg=amarillo!20!white}
\setbeamercolor{block title example}{use=structure,fg=white,bg=verde!80!black}
\setbeamercolor{block body example}{use=structure,fg=negro,bg=verde!20!white}
\setbeamercolor{block title alerted}{use=structure,fg=white,bg=rojo!90!negro}
\setbeamercolor{block body alerted}{use=structure,fg=negro,bg=rojo!20!white}
%%%%%%%%%%%%%%%%%%%%%FIN TEMA

%lstlisting %%%%%CODIGOS
\lstset{tabsize=4,language=C,basicstyle=\small,upquote=true,aboveskip={1.5\baselineskip},columns=fixed,showstringspaces=false, extendedchars=true,breaklines=true, prebreak = \raisebox{0ex}[0ex][0ex]{\color{gray75}{\ensuremath{\hookleftarrow}}},showtabs=false,showspaces=false,showstringspaces=false,identifierstyle=\ttfamily,keywordstyle=\color[rgb]{0,0,1}, commentstyle=\color[rgb]{0.133,0.545,0.133}, stringstyle=\color[rgb]{0.627,0.126,0.941}} 
  
%%%%%%%%%%%%%%PORTADA
\title[Titulo]{\textbf{\huge{Titulo}}\vspace{-0.3cm}}
\subtitle{subtitulo\vspace{-0.7cm}}
\author[Autor]{\small{\textbf{AUTOR1} \and \textbf{AUTOR2}\\ \texttt{Correo@ulagos.cl} \and \texttt{Correo@ulagos.cl}}\vspace{-0.2cm}}
\institute[ULA]{\small{\textbf{Departamento de Ciencias Exactas}\\Ingeniería Civil en Informática}\vspace{-0.5cm}}
\date[\today]{}
%%%%%%%%%%%%%FIN PORTADA

\begin{document}
\setbeamertemplate{background}{\includegraphics[height=9.2cm,width=12.8cm]{fondoula43}}%4:3
%\setbeamertemplate{background}{\centering\includegraphics[height=8.6cm,width=16.1cm]{fondoula169}}%16:9


%Pagina de Portada
\begin {frame} [plain]
\vspace{5.25cm}
\titlepage
\end {frame}
\setbeamertemplate{background}{}
%%%%%%%%%%%%%%%%ÍNDICES
\section[Contenido]{}
\frame{
  \frametitle{\textbf{Contenido}}
\setcounter{tocdepth}{2}%1: solo titulo principal, 2: titulo y subtitulo, 3....
\scriptsize
\tableofcontents[]
}%Generacion de Indice por capitulo
\AtBeginSection[]{
\begin{frame}
\frametitle{\textbf{Contenido}}
\scriptsize
\tableofcontents[currentsection]
\end{frame}}

\AtBeginSubsection[]{
\begin{frame}
\frametitle{\textbf{Contenido}}  
\scriptsize
\tableofcontents[currentsection,currentsubsection]
\end{frame}
}
%%%%%%%%%%%%%%%%FIN ÍNDICES







%%%%%%%%%%%%%%%%%%%%%%%%%%%%%%%%%%%%%
%%%%%%%%%%%%%%%%%INICIO PRESENTACION

\section{Prueba}
\begin{frame}[fragile]
\frametitle{\textbf{Colores y Bloques}}

 Lo siguiente son ejemplos de texto en colores: \rojo{Hola}, \verde{Hola}, \azul{hola}, \naranjo{hola}
 
 \begin{block}{Titulo}
Contenido
\end{block}


\begin{exampleblock}{Titulo ejemplo}
Contenido
\end{exampleblock}


\begin{alertblock}{Titulo alerta}
Contenido
\end{alertblock}
\end{frame}


\begin{frame}[fragile]
\frametitle{\textbf{Items, Descripciones y Enumeraciones}}
\justifying
\transdissolve
  \begin{itemize}
  \item Item sin numeros
    \begin{itemize}
  \item nivel 2
    \begin{itemize}
    \item nivel 3
    \end{itemize}
  \end{itemize}
\end{itemize}

\begin{enumerate}
  \item Item Numerado
    \begin{enumerate}
     \item Nivel 2
    \begin{enumerate}
      \item Nivel 3
    \end{enumerate}
  \end{enumerate}
\end{enumerate}

\begin{description}
  \item[Descripción] Texto descrito
  \begin{description}
  \item[Nivel 2] Texto
\end{description}

\end{description}
\end{frame}




\subsection{Códigos}
\begin{frame}[fragile]
\frametitle{\textbf{Código}}
   \vspace{-0.8cm}
\begin{lstlisting}
#include<stdio.h>
int main(){
 printf("Hola Mundo");
 return 0;
}
\end{lstlisting}\vspace{-0.3cm}
\end{frame}


\section{Matematica}
\begin{frame}[fragile]
\frametitle{\textbf{Ejemplo de Función Matemática}}
\justifying
 \begin{displaymath}
C_L=\frac{(S_{22}-\Delta S_{11}^*)^*}{|S_{22}|^2=-|\Delta|^2}
\end{displaymath}
    
\begin{displaymath}
R_S=\frac{\sqrt{1-g_s}\cdot (1-|S_{11}|^2)}{1-(1-g_s)\cdot|S_{11}|^2}
\end{displaymath}

\end{frame}





\section*{Bibliografía}
\begin{frame}[allowframebreaks]
\frametitle{\textbf{Bibliografía}}
\justifying
 \begin{thebibliography}{30}
 
%%%%%%LIBROS
\beamertemplatebookbibitems 
\bibitem {001} Davis, M.: Scientific Papers and Presentations. Academic Press, San Diego (1997)


%%%%ARTICULOS
\beamertemplatearticlebibitems
\bibitem {002} Day, R.A.: How to Write and Publish a Scientific Paper. Second edn. ISI Press, Philadelphia (1983)

\end{thebibliography}  
\end{frame}


%%%%%%%%%%%%%%%%%%%%%%%%%%%%%%
%%%%%%%%%%%%%%%%%FIN DOCUMENTO
\end{document}

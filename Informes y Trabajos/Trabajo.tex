\documentclass[letter,12pt]{report}
\usepackage[utf8]{inputenc}
\usepackage[T1]{fontenc}
\usepackage[spanish, es-tabla]{babel}
\usepackage[sfdefault, condensed]{roboto}
\usepackage[margin=1cm]{geometry}
\usepackage{multicol,graphicx,fancyhdr,eso-pic,url,float,cite,lmodern,listings,times,textcomp, amsthm,amsmath,amssymb,dsfont,color,colortbl,sidecap,xspace,epic,eepic,anysize,setspace, hyperref}
\usepackage{blindtext}

%%%%%%Glosario
%\usepackage[acronym]{glossaries}
%\makeglossaries
%\renewcommand{\glossaryname}{Glosario}
%\renewcommand{\acronymname}{Acrónimos}


%\usepackage{apacite} %bibliografias Bibtex

%Tipos de Letra
%\renewcommand{\rmdefault}{phv} % Arial
\usepackage{mathptmx} %Times
%Margenes
\marginsize{3cm}{2cm}{2cm}{2cm}
\spacing{1}%interlineado

  \providecommand{\keywords}[1]{\textbf{\textit{Palabras Clave---}} #1}

\newcommand\BackgroundPic{ \put(-3,0){ \parbox[b][\paperheight]{\paperwidth}{ \vfill \centering \includegraphics[width=\paperwidth,height=\paperheight]{portada.jpg} \vfill }}} 
  
  %Definicion de Colores
\definecolor{gray97}{gray}{.97}
\definecolor{gray75}{gray}{.75}
\definecolor{gray45}{gray}{.45}
\definecolor{verdeo}{rgb}{0,.5,0.2}
\definecolor{listinggray}{gray}{0.9}
\definecolor{lbcolor}{rgb}{0.9,0.9,0.9}
\newcommand\gris[1]{\textcolor[gray]{.35}{\emph{#1}}}
\newcommand\rojo[1]{\textcolor[rgb]{1,0,0}{#1}}
\newcommand\blue[1]{\textcolor[rgb]{0,0,1}{{#1}}}
\newcommand\azul[1]{\textcolor[rgb]{0,0,1}{#1}}
\newcommand\verde[1]{\textcolor[rgb]{0,.5,0.2}{#1}}
\newcommand\naranjo[1]{\textcolor[rgb]{1.00,0.36,0.06}{\textbf{#1}}}
\newcommand\blanco[1]{\textcolor[rgb]{1,1,1}{\textbf{#1}}}
\newcommand {\red}[1]{\textcolor[rgb]{1.00,0.00,0.00}{#1}}
\newcommand\cita[1]{{\scriptsize \begin{flushright}\emph{(#1)}\end{flushright}}}
  

  
%%%Entornos de desarrollo
\newtheorem{ejemplo}{Ejemplo}
\newtheorem{definir}{Definición}
\newtheorem{prueba}{Prueba}
\newtheorem{demo}{Demostración}
\newtheorem{obs}{Observación}

\newcommand{\ignore}[1]{}
  
  %%%CODIGOS DE PROGRAMACION
\lstset{%backgroundcolor=\color{lbcolor},
	frame=Ltb, framerule=0pt, aboveskip=0.5cm, tabsize=4, rulecolor=, language=C, %%%CAMBIAR POR LENGUAJE DE PREFERENCIA
 stringstyle=\ttfamily,  %basicstyle=\footnotesize,
        upquote=true, aboveskip={1.5\baselineskip}, columns=fixed, showstringspaces=false, extendedchars=true,breaklines=true, prebreak = \raisebox{0ex}[0ex][0ex]{\ensuremath{\hookleftarrow}}, showtabs=false, showspaces=false, showstringspaces=false,
        %tipos de letra y colores
        identifierstyle=\ttfamily,
        keywordstyle=\bfseries \color{black}, %palabras reservadas
        commentstyle= \scriptsize\color[RGB]{99,99,99}, %comentarios
        stringstyle=\color{black},%cadena de texto
        %numeracion de lineas
        framextopmargin=3pt, framexbottommargin=3pt, framexleftmargin=0.4cm,
        framesep=0pt, rulesep=.4pt, rulesepcolor=\color{black}, numbers=left, numbersep=15pt, numberstyle=\tiny, numberfirstline = false, breaklines=true,
}
%%%%FIN CODIGOS DE PROGRAMACION
\def\figurename{}
  
%%%%%%%%%%ENCABEZADO Y PIE DE PAGINA
%encabezado de las paginas pares e impares.
\rhead[PP]{Ing. Civil en Informática}
\renewcommand{\headrulewidth}{0.5pt}
%pie de pagina de las paginas pares e impares.
\lfoot[nombre]{Nombre Apellido}
\rfoot[rut]{Universidad de los Lagos}
\renewcommand{\footrulewidth}{0.5pt}
%encabezado y pie de pagina de la pagina inicial de un capitulo.
\fancypagestyle{plain}{
\fancyhead[R]{Ing. Civil en Informática}
\fancyfoot[L]{Nombre Apellido}
\fancyfoot[R]{Universidad de los Lagos}
\renewcommand{\headrulewidth}{0.5pt}
\renewcommand{\footrulewidth}{0.5pt}
}
\pagestyle{fancy} 
%%%%%%%%%%FIN ENCABEZADO Y PIE DE PAGINA 
 


\begin{document}

%%%%%%%%%%%PORTADA%%%%%%%%%%%%%%%%%%%%%
\setlength{\unitlength}{1 cm} %Especificar unidad de trabajo
\thispagestyle{empty}

\AddToShipoutPicture*{\BackgroundPic}

   \title{\scshape\Huge{Título del documento}\\\vspace{1cm}
        \Large Departamento de Ciencias Exactas\\
        \Large Ingeniería Civil en Informática\\
        \Large Asignatura\\
        \large Osorno, Chile}
   \author{
      Autor 1\\
      Correo 1
      \and Autor 2\\
      Correo 2
      \and Autor 3\\
      Correo 3
      \and Autor 4\\
      Correo 4
   }
   \date{\today}
   \maketitle
   \ClearShipoutPicture


\cleardoublepage
\pagenumbering{roman}
\setcounter{page}{1}

\tableofcontents
\listoffigures
%\renewcommand{\listtablename}{Índice de tablas}
\listoftables
%%%%%%%%%%%%%FIN PORTADA%%%%%%%%%%%%%%%%








%%%%%%%RESUMEN%%%%%%%%%%%
\begin{abstract}\thispagestyle{empty}

Resume en un (1) párrafo el contenido del informe en un máximo de 350 palabras.
Debe ser preciso:
\begin{itemize}\justifying
  \item Establece el problema
  \item Dice porqué es interesante
  \item Señala los logros y desafios
\end{itemize}
Un resumen debe ser llamativo, motivador, descriptivo y sin contenido específico. \textbf{No incluye}: citas, referencias, conclusiones, figuras ni tablas.



\keywords{Palabra1, Palabra2, Palabra3, Palabra4, Palabra5}
\end{abstract}

\cleardoublepage
\pagenumbering{arabic}
\setcounter{page}{1}






%%%%%%%%COMIENZO



\chapter{Generalidades}
Es lo que comúnmente se conoce como introducción, conduce al lector desde un tema de un área general hacia un campo de investigación específico, describe el contexto, el problema, motiva al lector.

Introduce la terminología, destaca las contribuciones del documento y da una breve descripción de la organización de éste.

Ejemplo de uso de una referencia \cite{001}. Ejemplo de referencia doble \cite{001,002}.

\section{Origen del Tema}
Contextualiza el trabajo respecto de investigaciones previas de otros autores y propias, señala las diferencias con trabajos previos. Algunas veces se incluye en la introducción o bien en la discusión del trabajo (secciones finales). Largo aproximado: 2 páginas.
\section{Planteamiento}

Provee un \naranjo{marco de referencia} para interpretar los resultados y conectarlos a la literatura existente sobre el fenómeno, orienta sobre cómo se realizará el estudio.

 Ayuda a prevenir errores que se han cometido en otros estudios, conduce al establecimiento de la hipótesis o afirmaciones que se someterán a prueba.
 
 Amplia el horizonte del estudio y centra al investigador en el problema, para evitar desviaciones del planteamiento original.

Considera una \naranjo{revisión bibliográfica} que consiste en detectar, obtener y consultar la bibliografía y otros materiales que pueden ser útiles para los propósitos del estudio.

La revisión bibliográfica debe ser selectiva, se puede realizar a partir de tres fuentes principales:

\begin{itemize}\justifying
  \item \naranjo{Primarias (directas):} Libros, artículos, antologías, tesis, disertaciones, entre otros.
  \item \naranjo{Secundarias:} Compilaciones, resúmenes de listados de referencias publicadas en un área en particular, bases de datos.
  \item \naranjo{Terciarias:} Documentos que reúnen nombres y títulos de revistas y otras publicaciones.
\end{itemize}


\begin{ejemplo}
\blindtext %reemplazar esta linea
\end{ejemplo} 

\section{Justificación y Aporte}
Justificar la conveniencia del proyecto desde diversos puntos de vista.

Preguntas clave:
  \begin{itemize}
  \item ¿Para qué sirve la investigación?
  \item ¿Quiénes se benefician con los resultados?
  \item ¿Ayuda a resolver algún problema práctico?
  \item ¿Contribuye a aumentar el conocimiento?
  \item ¿Se podrán generalizar los resultado?
\end{itemize}


\begin{ejemplo}
\blindtext %reemplazar esta linea
\end{ejemplo}


\section{Viabilidad}
Analizar la disponibilidad de recursos financieros, humanos y materiales.

Preguntas clave:
  \begin{itemize}\justifying
  \item ¿Puede llevarse a cabo esta investigación?
  \item ¿Cuánto tiempo tomará realizarla?
\end{itemize}




\chapter{Fundamentación y Justificación}
\section{Hipotesis}
Una \naranjo{hipótesis} es cualquier suposición, conjetura o predicción que se base en conocimientos existentes, en \naranjo{hechos nuevos} o en ambos.

 Propone una respuesta anticipada del problema, por lo que será aceptada o rechazada como resultado de la investigación.

\begin{itemize}
  \item No puede contradecir hechos conocidos y comprobados con anterioridad.
  \item Debe ser factible de comprobación.
  \item Tiene que ofrecer una explicación suficiente de los hechos o condiciones que pretende abarcar.
  \item \naranjo{Tiene que estar relacionada con el sistema de conocimiento correspondiente a los hechos que plantea el problema}.
\end{itemize}

Las hipótesis se pueden clasificar por:
 \begin{itemize}
  \item Contenido
  \item Forma
  \item Procesamiento estadístico
\end{itemize}

Hipótesis por Contenido
\begin{itemize}
  \item \textbf{Descriptivas}:
  \begin{itemize}
  \item Plantean supuestos sobre la estructura, las manifestaciones, funciones y características del objeto estudiado.
\end{itemize}

  \item \textbf{Causales o explicativas}:
  \begin{itemize}
  \item Plantean supuestos acerca de los vínculos de causa y efecto en el objeto estudiado que requieren comprobación experimental.
\end{itemize}

\end{itemize}


Hipótesis por Forma
\begin{itemize}
  \item \textbf{Generales}:
  \begin{itemize}
  \item Plantean supuestos que poseen un carácter generalizado del objeto de investigación.
\end{itemize}

  \item \textbf{Operacionales}:
  \begin{itemize}
  \item Plantean supuestos que en el desarrollo de la investigación tendrán carácter provisional hasta que se demuestre lo contrario.
\end{itemize}

\end{itemize}

Hipótesis por Procesamiento Estadístico
\textbf{Alternativa y nula}
 \begin{itemize}\justifying
  \item Utilizadas en estadística inferencial.
  \item Se plantean hipótesis mutuamente excluyentes: \naranjo{hipótesis nula} o \naranjo{hipótesis de investigación}.
  \item La hipótesis de investigación es una afirmación especial cuya validez se pretende demostrar, si las pruebas empíricas no la apoyan, se aceptará la hipótesis nula y se abandona la hipótesis.
\end{itemize}

\begin{ejemplo}
  Es posible optimizar la evaluación de consultas conjuntivas en bases de datos a través de índices.
\end{ejemplo}

\begin{ejemplo}
  Es posible crear un modelo de calidad de datos para portales web, que pueda ser usado para evaluar el nivel de calidad de los datos provistos por un portal web.
\end{ejemplo}

\section{Objetivos}\label{objetivos}

Se deben abordar desde el principio de la investigación, expresan los fines que se esperan lograr con el estudio del problema planteado, responden a la pregunta \naranjo{¿Para qué se lleva a cabo la investigación?}, por lo general comienzan con un verbo en infinitivo: Determinar, identificar, establecer, distinguir, medir, cuantificar, entre otros.

Deben enunciar un resultado unívoco, preciso, factible y medible. Su formulación debe ser clara, concisa y bien orientada hacia el fin, en función de ellos se plantean los métodos de recolección de datos, pruebas estadísticas, entre otros.

Evitar unir objetivos, idealmente, un objetivo general y varios específicos.

Cada objetivo específico se ``mapea'' a una pregunta de investigación.
Por ejemplo:
  \begin{itemize}
  \item \textbf{\naranjo{Objetivo:}} Optimizar los métodos de acceso a disco.
  \item \textbf{\naranjo{Preguntas de investigación:}} ¿Cuáles son los métodos de acceso a disco?
\end{itemize}
\subsection{General}
\blindtext %reemplazar esta linea

\subsection{Específicos}
\begin{enumerate}\justifying
  \item \blindtext %reemplazar esta linea
 
  \item \blindtext %reemplazar esta linea

\end{enumerate}

\section{Alcance}
Que se planea realizar y hasta que punto se espera llegar.

Esta subdivisión debe:
\begin{enumerate}\justifying
  \item Identifique el producto del software para ser diseñado por el nombre (por ejemplo, Anfitrión DBMS, el Generador del Reporte, etc.);
  \item Explique eso que el producto (del software hará y que no hará.
  \item Describe la aplicación del software especificándose los beneficios pertinentes, objetivos, y metas;
  \item Sea consistente con las declaraciones similares en las especificaciones de niveles superiores (por ejemplo, las especificaciones de los requisitos del sistema), si ellos existen.
\end{enumerate}




\section{Metodología}
Esto no es hacer referencia a métodos y herramientas que se usarán en el desarrollo del trabajo. Sino que describir como se llevará a cabo el trabajo.

Por lo tanto, nuevamente se puede plantear la solución (el proyecto) en términos explícitos de: los objetivos generales y específicos.

Posteriormente relacionar el cumplimiento de los objetivos específicos con tareas o actividades a desarrollar (al final se debe incluir seguramente actividades de validación y prueba del producto - plan de prueba).

\subsection{Planificación}
\blindtext %reemplazar esta linea

\subsection{Equipo de Trabajo}
\blindtext %reemplazar esta linea







En la Tabla \ref{t:info} se muestran las características de los sistemas GNU/Linux, obtenidas desde \cite{001}.


\begin{table}[hbt]
\begin{center}
\begin{tabular}{|l|p{10cm}|}\hline
\multicolumn{2}{|c|}{\textbf{Información general}}\\
\hline
\textbf{Modelo de desarrollo}&desarrollo	Software libre y código abierto\\
\textbf{Última versión estable}&Kernel: 4.11.3 (info) 25 de mayo de 2017 (10 días)\\
\textbf{Última versión en pruebas}&	4.12.rc2 (info) 22 de mayo de 2017 (13 días)\\
\textbf{Escrito en}&	C\\
\textbf{Núcleo}&	Núcleo Linux\\
\textbf{Plataformas soportadas}	& DEC Alpha, ARM, AVR32, Blackfin, ETRAX CRIS, FR-V, H8/300, Itanium, M32R, m68k, Microblaze, MIPS, MN103, PA-RISC, PowerPC, s390, S+core, SuperH, SPARC, TILE64, Unicore32, x86, Xtensa\\
\textbf{Licencia}	&GNU General Public License y otras\\
\textbf{Estado actual}	&En desarrollo\\
\textbf{En español}	&Sí\\
\hline
\end{tabular}
\end{center}
\caption{Información General de GNU/Linux}
\label{t:info}
\end{table}


\chapter{Cuerpo del documento}
\blindtext %reemplazar esta linea

\section{Introducción}
\blindtext %reemplazar esta linea

\section{Definición del Problema}
\blindtext %reemplazar esta linea

\section{Propuesta de Solución}
\blindtext %reemplazar esta linea



\chapter{Conclusión}
En las conclusiones se destaca lo mostrado en el trabajo, resaltando los resultados. Se indican los trabajos futuros. Usualmente, luego de las conclusiones se incluye un párrafo de agradecimientos a quienes auspician la investigación.
\section{Principales aportes}
\blindtext %reemplazar esta linea

\section{Contraste de resultados}
\blindtext %reemplazar esta linea






%%%%%
%agregar referencias
%\bibliographystyle{ieeetr}
%\bibliography{mybib.bib}

\begin{thebibliography}{}
%Bibliografía Formato IEEE
\bibitem {000} N. Apellido, Titulo, Revista, Edición, Paginas. Ciudad, Pais, Año,

\bibitem {001} R. M. Gutierrez, El impacto de la sobrepoblación de invertebrados en un ecosistema selvático, Revista Mundo Natural, 8, 73-82. 2013.

\bibitem {002} R.A. Day How to Write and Publish a Scientific Paper, Second edn. ISI Press, Philadelphia. 1983

\end{thebibliography}  












\renewcommand{\appendixname}{Anexos}
\appendix
\chapter{Anexos}
\section{Árbol de Problemas}
\blindtext %reemplazar esta linea
\section{Carta Gantt}
\blindtext %reemplazar esta linea

\section{Anexos del Trabajo}
\blindtext %reemplazar esta linea

\section{Anexo de ejemplo con código}

   \vspace{-0.8cm}
\begin{lstlisting}
-- Database: acuario

-- DROP DATABASE acuario;

CREATE DATABASE acuario
  WITH OWNER = postgres;


CREATE TABLE especies(
    sno integer PRIMARY KEY,
    nombre character varying(20),
    alimento character varying(20)
);

CREATE TABLE tanques(
    tno integer PRIMARY KEY,
    nombre_tanque character varying(20),
    color_tanque character varying(20),
    volumen  integer NOT NULL
);

CREATE TABLE peces(
    pno integer PRIMARY KEY,
    nombre_peces character varying(20),
    color_peces character varying(20),
    tno integer NOT NULL,
    sno integer NOT NULL,
    FOREIGN KEY (tno) REFERENCES tanques (tno) ON UPDATE CASCADE ON DELETE CASCADE,
    FOREIGN KEY (sno) REFERENCES especies (sno) ON UPDATE CASCADE ON DELETE CASCADE
);

CREATE TABLE eventos(
    eno integer PRIMARY KEY,
    pno integer NOT NULL,
    fecha date,
    FOREIGN KEY (pno) REFERENCES peces (pno) ON UPDATE CASCADE ON DELETE CASCADE
);



INSERT INTO especies VALUES(17,'delfin','arenque');
INSERT INTO especies VALUES(22,'tiburon','cualquier cosa');
INSERT INTO especies VALUES(74,'olomina','gusano');
INSERT INTO especies VALUES(93,'ballena','mantequilla de mani');
INSERT INTO especies VALUES(100,'pez espada','gusano');
INSERT INTO especies VALUES(120,'pez globo','gusano');

-- select * from especies

INSERT INTO tanques VALUES(55,'charco','verde',300);
INSERT INTO tanques VALUES(42,'letrina','azul',100);
INSERT INTO tanques VALUES(35,'laguna','rojo',400);
INSERT INTO tanques VALUES(85,'letrina','azul',100);
INSERT INTO tanques VALUES(38,'playa','azul',200);
INSERT INTO tanques VALUES(44,'laguna','verde',200);

-- select * from tanques


INSERT INTO peces VALUES (164, 'charlie', 'naranjo', 42, 74);
INSERT INTO peces VALUES (347, 'flipper', 'negro', 35, 17);
INSERT INTO peces VALUES (228, 'killer', 'blanco', 42, 22);
INSERT INTO peces VALUES (281, 'albert', 'rojo', 55, 17);
INSERT INTO peces VALUES (119, 'bonnie', 'azul', 42, 22);
INSERT INTO peces VALUES (388, 'cory', 'morado', 35, 93);
INSERT INTO peces VALUES (700, 'maureen', 'blanco', 44, 100);
INSERT INTO peces VALUES (800, 'beni', 'rojo', 55, 17);
INSERT INTO peces VALUES (900, 'nemo', 'rojo', 44, 74);
INSERT INTO peces VALUES (150, 'vicky', 'rojo', 55, 100);
INSERT INTO peces VALUES (160, 'mati', 'amarillo', 42, 100);
INSERT INTO peces VALUES (110, 'rafa', 'azul', 85, 100);
INSERT INTO peces VALUES (222, 'jimmy', 'amarillo', 38, 100);
INSERT INTO peces VALUES (144, 'bisho', 'rojo', 42, 93);
INSERT INTO peces VALUES (125, 'chris', 'azul', 38, 93);
INSERT INTO peces VALUES (183, 'sable', 'amarillo', 44, 93);
INSERT INTO peces VALUES (241, 'taz', 'rojo', 55, 93);
INSERT INTO peces VALUES (300, 'baltazar', 'azul', 85, 100);
INSERT INTO peces VALUES (200, 'cash', 'azul', 85, 100);
INSERT INTO peces VALUES (424, 'bandido', 'verde', 35, 100);
INSERT INTO peces VALUES (454, 'romo', 'blanco', 85, 93);


-- select * from peces

INSERT INTO eventos VALUES 
(3456 , 347 , '2010-01-26'),
(6653 , 164 , '2010-05-14'),
(5644 , 347 , '2010-05-15'),
(5645 , 347 , '2010-05-30'),
(6789 , 281 , '2010-04-30'),
(5211 , 228 , '2010-08-20'),
(6719 , 700 , '2010-10-22'),
(4555 , 164 , '2011-11-03'),
(9647 , 281 , '2011-12-06'),
(5347 , 281 , '2011-01-01');

--INSERT INTO eventos VALUES (3456, 164, '2010-01-26'); 
--INSERT INTO eventos VALUES (6653, 347, '2010-05-14'); 
--INSERT INTO eventos VALUES (5644, 347, '2010-05-15'); 
--INSERT INTO eventos VALUES (5645, 347, '2010-05-30'); 
--INSERT INTO eventos VALUES (6789, 228, '2010-04-30'); 
--INSERT INTO eventos VALUES (5211, 119, '2010-08-20'); 
--INSERT INTO eventos VALUES (6719, 388, '2010-10-22'); 
--INSERT INTO eventos VALUES (4555, 164, '2011-11-03'); 
--INSERT INTO eventos VALUES (9647, 281, '2011-12-21'); 
--INSERT INTO eventos VALUES (5369, 281, '2011-01-01'); 


-- ALTER TABLE tanques ADD medida character varying(2); 

-- UPDATE tanques SET medida = 'ml';

-- select * from tanques;

-- ALTER TABLE tanques DROP medida;

-- SELECT * FROM especies;
-- SELECT * FROM tanques;
\end{lstlisting}\vspace{-0.3cm}


\end{document}
